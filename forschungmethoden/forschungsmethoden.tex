\documentclass{../cssheet}

%--------------------------------------------------------------------------------------------------------------
% Basic meta data
%--------------------------------------------------------------------------------------------------------------

\title{Hinweise zu Forschungsmethoden}
\author{Prof. Dr. Christian Spannagel}
\date{\today}
\hypersetup{%
    pdfauthor={\theauthor},%
    pdftitle={\thetitle},%
    pdfsubject={Hinweise zu Forschungsmethoden},%
    pdfkeywords={literatur, artikel, publikationen, forschungsmethoden, methoden}
}

%\bibliographystyle{plain}
\nocite{*}
\defbibheading{references}[\refname]{\subsection{#1}
   \markboth{#1}{#1}}
   
   
%--------------------------------------------------------------------------------------------------------------
% document
%--------------------------------------------------------------------------------------------------------------

\begin{document}
\printtitle
\printdate
\singlespacing


\section{Grundlegende Hinweise}
\begin{itemize}
\item Die Servicestelle Forschungsmethoden der PH Heidelberg hat einen Moodle-Kurs zu Forschungsmethoden erstellt. Jede:r mit einem PH-Account kann sich direkt einschreiben: \url{https://moodle.ph-heidelberg.de/course/view.php?id=649}
\end{itemize}

\section{Literatur}
\printbibliography[keyword=scientificmethods,heading=references,title=Forschungsmethodik allgemein] 
\printbibliography[keyword=designresearch,heading=references,title=Design Research]
\printbibliography[keyword=qualanalysis,heading=references,title=Qualitative Inhaltsanalyse]
\printbibliography[keyword=quanti,heading=references,title=Quantitative Methoden]
\printbibliography[keyword=interview,heading=references,title=Interviews] 
\pagestyle{docstyle}

\vspace*{5mm}
\printlicense

\printsocials
\end{document}

