\documentclass{../cssheet}

%--------------------------------------------------------------------------------------------------------------
% Basic meta data
%--------------------------------------------------------------------------------------------------------------

\title{Rules for Tools}
\author{Prof. Dr. Christian Spannagel}
\date{\today}
\hypersetup{%
    pdfauthor={\theauthor},%
    pdftitle={\thetitle},%
    pdfsubject={Rules for the use of tools in my seminars},%
    pdfkeywords={tools, media, digitalization, ai, chatgpt, gpt3, socialmedia, twitter, linkedin, instagram, youtube, mastodon}
}


%--------------------------------------------------------------------------------------------------------------
% document
%--------------------------------------------------------------------------------------------------------------

\begin{document}
\printtitle
\begin{center}
Version 2.2 (October 1$^{st}$ 2023)
\end{center}
(German version on page~2)
\section {English Version}
\begin{enumerate}
\item \textbf{All media and tools are allowed.} You may use any media (texts, videos, \ldots) and tools (apps, calculators, \ldots) in my course that you find useful. This also applies to AI tools such as ChatGPT, which can be very helpful in generating ideas and writing texts, for example. So these tools are available to you in my course just as they are now in your everyday life and later in your job. By the way, the use of tools will also be subject of our course. Let's find out together how to use tools in a meaningful way to solve tasks! 
\item \textbf{You are responsible for your results}. All tools have their limitations. Information in media can be wrong. Calculators cannot work with real numbers. And AI language models like ChatGPT can produce well formulated texts, but they make erros and reproduce biases. So before you proceed with results and impulses, you need to check them and revise them if necessary. The tool does not think \emph{for you}, but \emph{you think with the help of the tool}. The theoretical background is the approach of \emph{distributed cognitions}: \inlinequote{Cognitions become `distributed' in the sense that the tool and its human partner think jointly. Whatever is produced is product of the joint system, resulting from the pooling together of the intelligences of both partners [\ldots]}{\parencite[p.~182]{salomon1993}}. In the end, however, you stand up for your solution. You have to be able to explain your solution to others. And you (not the tool) are responsible for errors in the solution. 
\item \textbf{Indicate aids used}. You must list all media you have used as sources. Likewise, you must list all tools used, except trivial tools such as pens or word processors. If applicable, also add how the tools were used (for example, writing down the prompt in the case of AI tools).
\item \textbf{Special rules apply to social media}. If you use social media (Instagram, TikTok, Mastodon, Twitter, \ldots) as part of my course and, for example, share your work results there, then you must follow some rules: You must, of course, observe the legal framework, such as copyright. You must treat others respectfully and follow netiquette. Other people may only be visible in your posts if they have given their permission (as far as I am concerned, I will give you permission if you follow the rules just mentioned). For links, you can find my social media accounts at the end of the document.
\item \textbf{No rules without exceptions}. If tools are not allowed in learning or assessment situations, I will let you know. I will also provide you with a reason for this.
\end{enumerate}

\newpage
\pagestyle{docstyle}

\section{German Version}

\begin{enumerate}
\item \textbf{Alle Medien und Werkzeuge sind erlaubt.} Sie dürfen sämtliche Medien (Texte, Videos, \ldots) und Werkzeuge (Apps, Taschenrechner, \ldots) in meiner Lehrveranstaltung verwenden, die Sie für sinnvoll halten. Dies gilt auch für KI-Werkzeuge wie ChatGPT, die zum Beispiel beim Generieren von Ideen und beim Verfassen von Texten sehr hilfreich sein können. Diese Hilfsmittel stehen Ihnen also in meiner Lehrveranstaltung genauso zur Verfügung wie jetzt im Alltag und später im Beruf. Die Nutzung von Hilfsmitteln wird übrigens auch Gegenstand in unserer Lehrveranstaltung sein. Lassen Sie uns gemeinsam herausfinden, wie man Werkzeuge sinnvoll für die Lösung von Aufgaben einsetzen kann! 
\item \textbf{Sie verantworten Ihre Arbeitsergebnisse.} Alle Hilfsmittel haben ihre Grenzen. Informationen in Medien können falsch sein. Taschenrechner können nicht mit reellen Zahlen rechnen. Und KI-Sprachmodelle wie ChatGPT können zwar gut formulieren, sie machen aber inhaltliche Fehler und reproduzieren Vorurteile. Bevor Sie also mit Ergebnissen und Impulsen weiterarbeiten, müssen Sie diese überprüfen und gegebenenfalls überarbeiten. Das Werkzeug denkt nicht \emph{für Sie}, sondern \emph{Sie denken mit Hilfe des Werkzeugs}. Theoretischer Hintergrund ist der Ansatz der \emph{verteilten Kognition}: \inlinequote{Cognitions become `distributed' in the sense that the tool and its human partner think jointly. Whatever is produced is product of the joint system, resulting from the pooling together of the intelligences of both partners [\ldots]}{\parencite[S.~182]{salomon1993}}. Am Ende stehen Sie aber für Ihre Lösung ein. Sie müssen Ihre Lösung anderen erklären können. Und für Fehler in der Lösung sind Sie verantwortlich (nicht das Werkzeug). 
\item \textbf{Geben Sie verwendete Hilfsmittel an.} Sie müssen alle Medien, die Sie bei einer Aufgabenlösung verwendet haben, als Quellen angeben. Ebenso müssen Sie alle verwendeten Werkzeuge aufführen, es sei denn es handelt sich um triviale Werkzeuge wie Kugelschreiber oder Textverarbeitung. Schreiben Sie gegebenenfalls auch dazu, wie die Werkzeuge verwendet wurden (zum Beispiel unter Angabe des Prompts bei KI-Tools).
\item \textbf{Für Social Media gelten spezielle Regeln.} Wenn Sie im Rahmen meiner Lehrveranstaltung Social Media (Instagram, TikTok, Mastodon, Twitter, \ldots) nutzen und zum Beispiel Ihre Arbeitsergebnisse darin teilen, dann müssen Sie dabei einige Regeln befolgen: Sie müssen natürlich die rechtlichen Rahmenbedingungen wie etwa das Urheberrecht beachten. Sie müssen respektvoll mit anderen umgehen und die Netiquette berücksichtigen. Andere Personen dürfen nur auf Ihren Posts zu sehen sein, wenn diese zugestimmt haben (Was meine Person angeht, erteile ich Ihnen die Erlaubnis, wenn Sie die eben genannten Regeln beachten). Für Verlinkungen finden Sie meine Social-Media-Accounts am Ende des Dokuments.
\item \textbf{Keine Regeln ohne Ausnahmen.} Wenn in Lern- oder Prüfungssituationen Hilfsmittel nicht erlaubt sind, dann teile ich Ihnen das mit. Ich liefere Ihnen auch eine Begründung dafür.
\end{enumerate}

\newpage
\printbibliography[keyword=rulesfortools,title=References]

\vspace*{10mm}
\printlicense

\printsocials

\end{document}
