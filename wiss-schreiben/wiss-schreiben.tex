\documentclass{../cssheet}

%--------------------------------------------------------------------------------------------------------------
% Basic meta data
%--------------------------------------------------------------------------------------------------------------

\title{Verfassen wissenschaftlicher Texte}
\author{Prof. Dr. Christian Spannagel}
\date{\today}
\hypersetup{%
    pdfauthor={\theauthor},%
    pdftitle={\thetitle},%
    pdfsubject={Hinweise zum wissenschaftlichen Schreiben},%
    pdfkeywords={literatur, artikel, publikationen, wissenschaft, schreiben}
}

%--------------------------------------------------------------------------------------------------------------
% document
%--------------------------------------------------------------------------------------------------------------

\begin{document}
\printtitle
\printdate
\singlespacing

\tableofcontents

\section{Vorwort}

Studierende werden oft zum ersten Mal während Ihres Studiums mit der
Situation konfrontiert, einen wissenschaftlichen Text schreiben zu
müssen. Hierzu zählt vor allem das Schreiben einer Abschlussarbeit wie
der Bachelorarbeit oder der Masterarbeit, aber auch das
Schreiben von Ausarbeitungen zu Referaten oder von Unterrichtsentwürfen
fällt in diese Kategorie. In der Regel herrscht bei den Betroffenen
bezüglich der Regeln und Konventionen, die beim Schreiben einer solchen
Arbeit beachtet werden müssen, eine große Unsicherheit. Der vorliegende
Text möchte einen Beitrag dazu leisten, diese Unsicherheit abzubauen.

Die Konventionen und Regeln, die hier beschrieben werden, sind sicher
nicht all\-ge\-mein\-gül\-tig. Genau wie über Geschmack lässt sich auch über
Empfehlungen zum Verfassen von Texten streiten. Die Hinweise hier sind
auch nicht verpflichtend gemeint. Du darfst beispielsweise auch gerne
andere Zitationsweisen o.\,ä. verwenden, wenn du diese für vernünftig
hältst. Hauptsache ist, dass du konsistent bei dem bist, was du tust. Du
kannst aber jederzeit einen Blick hier in die Hinweise werfen, wenn du
eine Frage hast. Auf viele Fragen gibt es hier nämlich Antworten.

Bitte beachte auch, dass es sich um Empfehlungen von mir an Studierende handelt, die bei mir einen wissenschaftlichen Text schreiben. Andere Dozent:innen haben andere Anforderungen und Vorstellungen. Sprich also in jedem Fall mit deinem:deiner Dozent:in und verlasse dich nicht einfach auf diesen Text.

Der vorliegende Text wurde nach den in ihm beschriebenen Konventionen
verfasst und hat somit einen -- zumindest in seiner äußeren Form --
exemplarischen Charakter. Lediglich die humorvolle Art entspricht nicht
der eines wissenschaftlichen Texts. Im Gegensatz hierzu ist in
wissenschaftlichen Arbeiten eine gewisse Ernsthaftigkeit geboten. (In
wissenschaftlichen Arbeiten findet man hin und wieder trotzdem
humorvolle Passagen. Auch Wissenschaftler:innen lachen gelegentlich.)

Am Anfang einer wissenschaftlichen Arbeit sollte dem:der Leser:in ein grober
Überblick über den Text geboten werden. Dieser Text soll dabei keine
Ausnahme sein. Im nächsten Abschnitt werden einige Empfehlungen zu Stil
und Ausdruck gegeben. Der darauf folgende Abschnitt enthält Hinweise zum Aufbau
und zur Struktur eines Texts. Dann werden Konventionen zur Form des
Texts dargestellt. Das Belegen von Aussagen und das Zitieren
wissenschaftlicher Literatur ist eine Wissenschaft für sich. Daher ist
dieser ein eigener Abschnitt gewidmet. Nähere Auskünfte zum
Literaturverzeichnis schließen nahtlos an. Der Text schließt mit einigen
allgemeinen, vernünftigen Hinweisen.

Das Dokument wird ständig überarbeitet. Für Hinweise und Verbesserungsvorschläge bin ich dankbar. 

\pagestyle{docstyle}
%\tableofcontents


\section{Stil und Ausdruck}

\subsubsection*{Schreibe in kurzen Sätzen.}

Lange Sätze, die durch Relativsätze, wie dieser hier einer ist, oder
durch andere Konstruktionen, beispielsweise eingeschobene Sätze,
unterbrochen sind, können Leser:innen, die versuchen, die Sätze zu
verstehen, in arge Bedrängnis bringen, da sie bei dem Unterfangen, die
Struktur der Sätze zu durchschauen, den Überblick verlieren könnten.

Ein Beispiel von Eco (1993, S. 186): \glqq{}Schreibt nicht: Der Pianist
Wittgenstein, der Bruder des bekannten Philosophen, der den
Tractatus-Logico-Philosophicus schrieb, den viele für das Hauptwerk der
zeitgenössischen Philosophie halten, hatte das Glück, dass Ravel für ihn
das Konzert für die linke Hand schrieb, weil er seine rechte im Krieg
verloren hatte. Schreibt lieber: Der Pianist Wittgenstein war der Bruder
des Philosophen Ludwig. Da er im Krieg die rechte Hand verloren hatte,
schrieb Ravel für ihn das Konzert für die linke Hand.\grqq{} Ich würde Ecos
Vorschlag noch optimieren wollen. Schreibe noch lieber: \glqq{}Der Pianist
Wittgenstein war der Bruder des Philosophen Ludwig. Er hatte im Krieg
die rechte Hand verloren. Daher schrieb Ravel für ihn das Konzert für
die linke Hand.\grqq{} Das hat eine zwingendere Logik. Also nicht: \glqq{}Die Straße
ist nass, weil es geregnet hat.\grqq{} (Das ist chronologisch falsch herum.)
Sondern: \glqq{}Es hat geregnet. Daher ist die Straße nass.\grqq{}

Schreibe tendenziell in Hauptsätzen, die maximal von einem Nebensatz
begleitet sind. Vor allem schwierige Gedankengänge lassen sich in
einfachen Sätzen verständlicher vermitteln.

\subsubsection*{Meide Wiederholungen.}

Meide Wiederholungen. Suche nach Alternativen, wenn du dasselbe Wort
oder dieselbe Wendung in räumlicher Nähe mehrfach verwendest. Meide
Wiederholungen.

\subsubsection*{Vermeide substantivierte Verben.}

Die Sprache wird lebendiger, wenn du Verben anstelle von
substantivierten Verben verwendest. Also nicht: \glqq{}Eine Erhöhung der
Lebendigkeit der Sprache wird durch die Vermeidung der Substantivierung
von Verben erzielt.\grqq{} Damit kannst du allenfalls Verwaltungsbeamt:innen oder
Jurist:innen beeindrucken.


\subsubsection*{Verwende Fremdwörter in Maßen.}

Du musst nicht unbedingt \emph{echauffieren} sagen, wenn du
\emph{empören} meinst. Und \emph{beziehungsweise} klingt auch nicht so
arrogant -- Entschuldigung: überheblich - wie \emph{respektive}.
Fachwörter sind von dieser Regel natürlich ausgenommen.

\subsubsection*{Vermeide doppelte Verneinungen.}

Doppelte Verneinungen sind nicht selten schwer zu verstehen. Statt
\glqq{}nicht ungünstig\grqq{} solltest du \glqq{}günstig\grqq{} schreiben -- natürlich nur, wenn
es nicht unvermeidbar ist.

\subsubsection*{Schreibe nicht in der ersten Person.}

Vermeide die Verwendung von \emph{ich} und \emph{wir}. Viele Leser:innen wissenschaftlicher
Texte halten dies für unangemessen. Schreibe also nicht: \glqq{}Ich habe nach
einer umfangreichen Recherche herausgefunden, dass man die erste Person
nicht verwenden sollte.\grqq{} Sondern: \glqq{}Eine umfangreiche Recherche hat
ergeben, dass man die erste Person nicht verwenden sollte.\grqq{}

In seltenen Fällen, beispielsweise in Fußnoten, kann man \glqq{}der Autor\grqq{}
oder \glqq{}der Verfasser\grqq{} verwenden, also beispielsweise: \glqq{}Anmerkung des
Autors\grqq{}.

Anmerkung: Diese Regel ist umstritten. Manche
Personen setzen sich geradezu für die Verwendung von \emph{ich} und \emph{wir} ein.
Wenn man die erste Person verwendet, dann sollte man sich aber im Klaren
sein, dass einige Leser:innen dies für seltsam halten und den Text unter
Umständen deshalb nicht ernst nehmen werden.

\subsubsection*{Formuliere möglichst im Aktiv und nicht im Passiv.}

Wenn man die erste Person meidet, dann besteht die Gefahr, dass man
stattdessen das Passiv verwendet oder auf das Wort \emph{man} zurückgreift.
Aktiv-Formulierungen sind allerdings viel lebhafter und wirken nicht so
eingestaubt wie Passiv-Konstruktionen. Wenn du also
nicht schreiben möchten \glqq{}Wir haben herausgefunden, dass...\grqq{}, dann
schreibe weder \glqq{}Es wurde herausgefunden, dass...\grqq{} noch \glqq{}Man hat
herausgefunden, dass...\grqq{}, sondern beispielsweise \glqq{}Die Datenanalyse hat
ergeben, dass...\grqq{} oder \glqq{}Die Interpretation der Ergebnisse lässt den
Schluss zu, dass...\grqq{}.

Natürlich darf man auch hin und wieder Passivkonstruktionen verwenden.
Das Passiv ist ja schließlich im Deutschen nicht verboten und seine
Verwendung steht auch nicht unter Strafe. Entscheidend ist aber, dass es
nicht zum \emph{stilprägenden} Element dadurch wird, dass man es ständig
verwendet.

\subsubsection*{Verwende gender-gerechte Sprache.}

Einleitende Fußnoten wie beispielsweise \glqq{}In diesem Text wird immer die männliche Form verwendet. Die weibliche Form ist aber immer mit gemeint.\grqq{} gehen gar nicht. Das \emph{generische Maskulinum} sollte nicht verwendet werden, insbesondere wenn sich dadurch Stereotype manifestieren könnten. Beispielsweise sollte man gerade in den Fächern Mathematik und Informatik auf eine ordentliche gendergerechte Schreibweise achten, weil diese Fächer ein sehr maskulines Image haben. Schreibe also \glqq{}Schülerinnen und Schüler\grqq{}, wenn du Schülerinnen und Schüler meinst. 

Du solltest allerdings nicht im selben Satz von \glqq{}Schülerinnen und
Schülern und Lehrerinnen und Lehrern\grqq{} sprechen -- das zerstört die
Lesbarkeit. Alternativ kannst du beispielsweise auch \glqq{}die Lernenden\grqq{},
\glqq{}die Lehrenden\grqq{} sowie \glqq{}die Lehrpersonen\grqq{} verwenden.

Mittlerweile sind auch die Verwendung von Gender-Platzhaltern wie Sternchen oder Doppelpunkt üblich, also Schüler*innen, Schüler:innen oder Schüler\_innen. In diesem Text wird die Schreibweise mit dem Doppelpunkt verwendet, wie du vielleicht schon bemerkt hast.

Schreibe unter keinen Umständen \glqq{}die SchülerInnen\grqq{} oder \glqq{}die
Schüler/-innen\grqq{}, es sei denn, du möchtest deinen Text mit
Rechtschreibfehlern versehen oder ihn unleserlich gestalten. 

\subsubsection*{Wichtige Vokabeln}

Wichtige Vokabeln für dich sind: Klarheit, Einfachheit, Prägnanz, Kürze,
Präzision, Exaktheit und Strukturiertheit. Aufgabe: Ergänze die Liste
mit weiteren wichtigen Vokabeln.

\section{Aufbau und Struktur}

\subsubsection*{Titel}

Dein Titel sollte kurz und aussagekräftig sein. Es kommt hin und wieder
vor, dass Personen bereits durch ihren Titel die Hälfte der erlaubten
Wortanzahl verbraucht haben. Das sollte dir nicht passieren -- aus mehr
als 15 Wörtern sollte der Titel nicht bestehen. Darüber hinaus sollte
der Titel deines Texts reizvoll klingen, aber dennoch wissenschaftlich.
Eine Gratwanderung, wie du bestimmt schon vermutest.

\subsubsection*{Teile den Leser:innen am Anfang mit, worum es geht.}

Schreibe am Anfang klar und deutlich, worum es in dem Text geht.

Wenn du einen langen Text verfasst (z.\,B. Bachelorarbeit oder eine Masterarbeit), dann beschreibe am besten in einem Einleitungskapitel kurz
und bündig die Zielsetzung deiner Arbeit. Es empfiehlt sich auch, am
Ende des Einleitungskapitels einen Überblick über die Arbeit zu geben
(\glqq{}In Kapitel~2 wird ein Überblick über die theoretischen Grundlagen
gegeben. Kapitel~3 enthält ...\grqq{}).

Wenn du einen kurzen Text verfasst (z.\,B. einen wissenschaftlichen
Artikel), dann schreibe am Anfang eine Kurzfassung (\emph{Abstract}), in der
die Arbeit kurz zusammengefasst wird und die zentralen Ergebnisse
genannt werden. Die Kurzfassung beantwortet die folgenden Fragen: Was
wurde getan? Was wurde herausgefunden? Was sind die Schlussfolgerungen?
Scheue dich nicht, die Ergebnisse deiner Arbeit tatsächlich in eins,
zwei Sätzen im Abstract zusammenzufassen. Denke besser nicht, die Leser:innen
würden ganz gespannt auf die Ergebnisse sein, wenn du am Anfang ein
Geheimnis daraus machst. Das Gegenteil ist der Fall: Wissenschaftler:innen
entscheiden mitunter am Abstract, ob sie einen Text des Lesens für wert
befinden oder nicht. Sie lesen den Text vielleicht, wenn die Ergebnisse
interessant zu sein scheinen, und deshalb müssen sie diese auch aus dem
Abstract entnehmen können. Und: Sollten diese Wissenschaftler:innen den Text
dennoch nicht lesen, dann kennen sie trotzdem die zentralen Ergebnisse.
Das ist doch was.

Die Kurzfassung steht direkt unter dem Titel und dem:der Autor:in der Arbeit
und vor dem ersten Kapitel. Sie ist in der Regel links und rechts leicht eingerückt.

\subsubsection*{Hamster IMRaD}

Für wissenschaftliche Artikel, in denen empirische Studien beschrieben
werden, bietet sich das IMRaD-Schema an. IMRaD steht
für \emph{Introduction, Methods, Results, and Discussion}. In Kürze erläutert:

\begin{itemize}
\item
  Introduction: Beschreibe hier, welche Fragestellung untersucht wurde
  und warum. In der Introduction muss den Leser:innen klar
  werden, weshalb es sinnvoll ist, dass du diese Studie durchgeführt
  hast.
\item
  Methods: Beschreibe hier, mit welchen Methoden du die Fragestellung
  untersucht hast. Oft enthält dieser Abschnitt noch Unterabschnitte wie
  \emph{Versuchspersonen} oder \emph{Stichprobe}, \emph{Materialien} und
  \emph{Versuchsaufbau}.
\item
  Results: In diesem Abschnitt werden die Ergebnisse der Studie
  aufgeführt. Aber Vorsicht: Hier wird noch nicht interpretiert! Es
  handelt sich hingegen um eine mehr oder weniger nüchterne Auflistung
  der (statistischen) Ergebnisse. Deine begründete Interpretation der
  Ergebnisse gibt's erst im nächsten Abschnitt.
\item
  Discussion: Hier darfst du interpretieren -- natürlich nicht aus der
  Luft gegriffen, sondern immer schön begründet.
\end{itemize}

Letztlich gibt das IMRaD-Schema die Struktur von Artikeln nur grob
wieder. Beispielsweise findet man oft zwischen \emph{Introduction} und
\emph{Methods} noch einen Theorieteil (dieser empfiehlt sich, wenn man dir
nicht vorwerfen soll, du hättest theoriefrei gearbeitet). Ganz am Ende
gibt's oft noch eine Zusammenfassung und einen Ausblick (\emph{Conclusion and
Future Work}).

\subsubsection*{Sanduhren}

Die inhaltliche Ausgestaltung deines Texts sollte der Form einer Sanduhr
entsprechen: Zu Beginn (Einleitung) sollte dein Text breit angelegt
sein, den Forschungsstand beschreiben, das Forschungsfeld
charakterisieren. Dann, im Empirie-Teil (Methods, Results, Discussion),
wird's eng. Du beschreibst deine spezielle (eng angelegte) Studie im
Detail. Am Ende wird's wieder breit: In der Zusammenfassung erläuterst
du die Bedeutung deiner Arbeit im Forschungsfeld, welche allgemeinen
Schlussfolgerungen sich (vorsichtig) aus den Ergebnissen ziehen lassen,
und was man sonst so noch in Zukunft machen könnte.

\subsubsection*{Inhalts-, Abbildungs- und Tabellenverzeichnis}

Diese Verzeichnisse benötigst du nur bei längeren Texten (z.\,B.
Bachelorarbeiten oder Masterarbeiten). Das Inhaltsverzeichnis, das
Abbildungs- und das Tabellenverzeichnis befinden sich am Anfang des
Texts.

\subsubsection*{Einteilung in Kapitel bzw. Abschnitte}

Die Einteilung in Kapitel und Unterabschnitte verdeutlicht die Struktur
eines Texts. Jeder Abschnitt wird durch eine passende Überschrift
eingeleitet.

\subsubsection*{Anhang}

Der Anhang ist ein praktischer Ort, in dem größere Elemente wie
komplette Fragebogen, Arbeitsergebnisse von Schüler:innen oder ähnliches
abgelegt werden können. Jedes Element im Anhang erhält eine eigene
Überschrift. Die Nummerierung der Überschriften erfolgt auf oberster
Ebene in großen lateinischen Buchstaben, auf tieferen Ebenen wird mit
Zahlen nummeriert. Beispiele:

\begin{itemize}
\item
  Anhang A: Fragebogen zur Lernmotivation
\item
  Anhang B.1: Auszüge aus dem Lerntagebuch von Miriam
\item
  Anhang B.2: Auszüge aus dem Lerntagebuch von Max
\end{itemize}

\section{Form}

\subsubsection*{Mache dich über die Anforderungen kundig.}

Wenn du eine Bachelor- oder Masterarbeit schreibst, dann erkundige dich beim Prüfungsamt und deinem:deiner Prüfer:in über die offiziellen Vorgaben. Wenn du einen Artikel für eine Zeitschrift oder ein Buch schreibst, dann wirst du von den Herausgeber:innen entsprechend informiert.

\subsubsection*{Verwende nur eine einzige Schriftart. Sei sparsam mit Schriftmodi.}

Wenn du die Wahl hast, dann entscheide dich für eine Serifenschrift --
am besten Times New Roman (12~Punkt). Sei außerdem sparsam mit Schriftmodi.
Schreibe Hervorhebungen \emph{kursiv}, und verzichte auf Fettdruck und
Unterstreichungen.

\subsubsection*{Seitenränder und Zeilenabstand}

Bei einer Bachelor- oder Masterarbeit verwendet man oftmals einen Zeilenabstand von 1,5. Die Seitenränder sollten mindestens 2~cm betragen, der linke Seitenrand hingegen sollte größer sein (mindestens 2,5~cm), wenn die Arbeit gebunden werden soll.

\subsubsection*{Überschriften}

Kapitelüberschriften werden nummeriert. Beispiele:

\begin{itemize}
\item
  2. Theoretische Grundlagen
\item
  2.1 Lerntheorien
\item
  2.1.1 Konstruktivismus
\end{itemize}

Es ist zu beachten, dass auf der obersten Ebene die Nummerierung mit
einem Punkt abgeschlossen wird (\glqq{}2. Theoretische Grundlagen\grqq{}), auf den
unteren Ebenen aber nicht (\glqq{}2.1 Lerntheorien\grqq{}). Also nicht: \glqq{}2.1.
Lerntheorien\grqq{}.

Ebenso sollten sich die Überschriften auf unterschiedlichen Ebenen durch
ihr Erscheinungsbild unterscheiden. Eine beispielhafte Formatierung (bei
einem Text in 12-Punkt-Schrift) ist:

\begin{itemize}
\item
  1. Ebene: 14 Punkt, fett
\item
  2. Ebene: 12 Punkt, fett
\item
  3. Ebene: 12 Punkt, kursiv
\end{itemize}

Versuche auf die Einführung einer vierten Ebene zu verzichten.

\subsubsection*{Nummeriere und beschrifte Abbildungen und Tabellen.}

Nummeriere Abbildungen und Tabellen. Zähle für Abbildungen und Tabellen
getrennt, d.h. \emph{Abbildung 1}, \emph{Abbildung 2}, \emph{Abbildung 3} und \emph{Tabelle
1}, \emph{Tabelle 2} und \emph{Tabelle 3}.

Erwähne jede Abbildung und jede Tabelle im Text. Beispiele:

Das Arbeitsgedächtnis besteht aus mehreren Komponenten (Abbildung 4).
Tabelle 3 zeigt die Mittelwerte und Standardabweichungen für alle
Versuchsgruppen

Gib Abbildungen und Tabellen eine Beschriftung. Diese befindet sich
unterhalb von Abbildungen, aber oberhalb von Tabellen. Beispiele:

Abbildung 4: Die Struktur des Arbeitsgedächtnisses

Tabelle 3: Mittelwerte und Standardabweichungen

\subsubsection*{Verwende Fußnoten sparsam.}

Eine große Anzahl von Fußnoten stört den Textfluss.

\subsubsection*{Besondere Zeichen}

Gedankenstrich (--) und Bindestrich (-) sind unterschiedliche Zeichen. Zwischen Seitenzahlen steht immer ein Gedankenstrich, also z.B. \glqq{}S. 45--57\grqq{}, auch im Literaturverzeichnis!

In deutschsprachigen Texten solltest du auch deutsche Anführungzeichen
verwenden, also unten beginnend und oben endend: \glqq{}Text\grqq{}.

\section{Belegen und zitieren}

\subsubsection*{Belege Aussagen mit passenden Literaturstellen.}

Belege Aussagen und Erkenntnisse anderer mit entsprechenden Vermerken.
Sie stützen dadurch deine Aussagen. Belege verweisen auf
Literaturangaben am Ende des Texts (siehe unten). Es gibt verschiedene
Richtlinien, wie Belege zu schreiben sind. Nach den Richtlinien der
American Psychological Association (APA, 2020) stehen sie am Ende der zu
belegenden Aussage in Klammern und bestehen aus dem Namen des:der Autor:in
oder der Autor:innen, einem Komma und der Jahreszahl des Werks. Der Beleg
kommt in der Regel ans Ende des Satzes und steht dabei vor dem Punkt.
Beispiel:

\begin{itemize}
\item
  Das Kurzzeitgedächtnis ist stark kapazitätsbegrenzt (Miller, 1956).
\end{itemize}

Mehrere Belege werden innerhalb der Klammer mit einem Semikolon
getrennt. Beispiel:

\begin{itemize}
\item
  Das Kurzzeitgedächtnis ist stark kapazitätsbegrenzt (Miller, 1956;
  Chandler \& Sweller, 1991).
\end{itemize}

Wird der:die Autor:in im Text genannt, dann steht hinter dem Autorennamen die
Jahreszahl in Klammern. Sein:ihr Name wird nicht doppelt genannt.
Beispiel:

\begin{itemize}
\item
  Nach Miller (1956) ist das Kurzzeitgedächtnis stark
  kapazitätsbegrenzt.
\end{itemize}

So wär's \emph{zu umständlich}: \glqq{}Nach Miller ist das Kurzzeitgedächtnis
stark kapazitätsbegrenzt (Miller, 1956).\grqq{} Und so bitte auch
\emph{nicht}: \glqq{}Nach Miller (Miller, 1956) ist das Kurzzeitgedächtnis
stark kapazitätsbegrenzt.\grqq{}

Will man einen größeren Abschnitt mit einer Literaturstelle belegen, so
fügt man den Beleg selbstverständlich nicht hinter jedem Satz ein. Am
Ende des Abschnitts kommt der Beleg aber in der Regel zu spät. Eine gute
Lösung für diesen Fall ist die einmalige Erwähnung des Belegs im ersten
Satz des Absatzes bzw. an dessen Ende. Der weitere Absatz sollte dann so
fortgeführt werden, dass den Leser:innen klar wird, dass sich der Beleg auf
den gesamten Abschnitt bezieht. Beispiel:

\begin{itemize}
\item
  Nach Baddeley (1992) besteht das Arbeitsgedächtnis aus mehreren
  Komponenten. In seinem Modell sind die phonologische Schleife und der
  räumlich visuelle Notizblock miteinander über die zentrale Exekutive
  verbunden. Die phonologische Schleife\ldots{} (jetzt folgt ein
  größerer Abschnitt, in dem Baddeleys Modell beschrieben wird).
\end{itemize}

Haben mehr als zwei und weniger als sechs Autor:innen ein Werk geschrieben,
so werden bei der ersten Nennung alle Autor:innen genannt, bei jeder
weiteren nur der:die Erstautor:in, gefolgt von \glqq{}et al.\grqq{} (\glqq{}et alii\grqq{}, lateinisch
für \glqq{}und andere\grqq{}). Beispiel:

\begin{itemize}
\item
  Bei der ersten Nennung werden alle Autor:innen genannt (Mandl, Gruber \&
  Renkl, 2002).
\item
  Ab der zweiten Nennung nicht mehr (Mandl et al., 2002).
\end{itemize}

Bitte beachte, dass der Punkt nicht hinter dem \glqq{}et\grqq{} steht, sondern
hinter dem \glqq{}al\grqq{}, denn das wird ja abgekürzt.

Haben sechs oder mehr Autor:innen ein Werk geschrieben, so wird gleich bei
der ersten Nennung nur der:die Erstautor:in mit \glqq{}et al.\grqq{} genannt.

Beim Kürzen kann es geschehen, dass zwei unterschiedliche Werke
plötzlich denselben Verweis erhalten. In diesem Fall sind so viele
Autorennamen anzuführen, dass die Verweise unterscheidbar sind.

Das Kaufmanns-Und (\&) wird nur verwendet, wenn die Namen innerhalb der
Klammer stehen. Beispiele:

\begin{itemize}
\item
  Im situierten Lernen wird den Lernenden eine aktive Rolle zugewiesen
  (Mandl, Gruber \& Renkl, 2002).
\end{itemize}

Aber:

\begin{itemize}
\item
  Nach Mandl, Gruber und Renkl (2002) lassen sich die folgenden
  Charakteristika situierten Lernens benennen\ldots{}
\end{itemize}

Wenn du mehrere Werke derselben Autor:inn:en aus demselben Jahr zitierst,
dann füge Kleinbuchstaben an die Jahreszahlen an, also z.B.: (Salomon,
1993a) und (Salomon, 1993b). Im Literaturverzeichnis müssen die Werke
ebenfalls auf diese Weise gekennzeichnet sein.

Kör\-per\-schafts\-au\-to\-ren werden auf folgende Weise angeführt: (Eu\-ro\-päi\-sche Kommission, 1995). Ist der Körperschaftsname lang, so genügt eine
Abkürzung, wenn diese im Literaturverzeichnis entsprechend genannt ist.
Beispiel:

\begin{itemize}
\item
  Die Kompetenzen, die im Bildungsplan 2016 des Landes Baden-Württemberg
  für die Sekundarstufe I (MKJS, 2016) ausgeführt sind, werden\ldots{}
\end{itemize}

Und hier die dazugehörige Literaturangabe im Literaturverzeichnis:

\begin{itemize}
\item
  MKJS -- Ministerium für Kultus, Jugend und Sport Baden-Württemberg (Hrsg.) (2016). \emph{Bildungsplan 2016 für die Sekundarstufe I}. \url{http://www.bildungsplaene-bw.de/,Lde/Startseite} (Stand: 20. Januar 2006).
\end{itemize}

Belege stehen nur in Ausnahmefällen in Fußnoten, z.B. wenn man zur
Quelle noch etwas anmerken möchte. Belege enthalten in der Regel auch
keine Seitenzahlen. Seitenzahlen werden im Allgemeinen nur bei Zitaten
hinzugefügt.

Generell gilt: Alle Belege müssen im Literaturverzeichnis am Ende des
Texts aufgeführt sein!

\subsubsection*{Zitiere Aussagen, die in deinem Zusammenhang von zentraler Bedeutung
sind.}

Bislang wurden nur einfache Belege angesprochen. Die Beispiele im
vorangegangenen Abschnitt sind Aussagen \emph{in eigener Formulierung},
die mit Werken anderer Autor:innen belegt werden. Man kann
selbstverständlich andere Autor:innen auch wörtlich wiedergeben. Dann
handelt es sich aber um \emph{Zitate}. Wann man nur belegt und wann man
zitiert, ist schwierig zu sagen. Eine grobe Richtlinie ist, dass man
Aussagen prinzipiell selbst formuliert und mit Literaturstellen belegt
und nur selten wortgetreu zitiert. In der Regel zitiert man
beispielsweise die Definition eines Begriffs oder grandios formulierte
Aussagen, die man selbst nicht hätte besser treffen können.

Kurze Zitate werden in den Fließtext eingebaut und mit Anführungszeichen
umgeben. Am Ende des Zitats wird der Beleg inklusive Seitenzahl
angegeben. Beispiel: \glqq{}Pure discovery did not work in the 1960s, it did
not work in the 1970s, and it did not work in the 1980s, so after these
three strikes, there is little reason to believe that pure discovery
will somehow work today.\grqq{} (Mayer, 2004, S. 18) Längere Zitate werden in
einem eigenen Absatz links und rechts eingerückt dargestellt und ohne
Anführungszeichen geschrieben.

Vor dem Seitenhinweis steht ein Komma und ein Leerzeichen, nach \glqq{}S.\grqq{}
folgt ebenfalls ein Leerzeichen.

Auslassungen in Zitaten werden mit [\ldots] gekennzeichnet. Alte
Rechtschreibung in Zitaten bleibt bestehen und wird nicht an die neue
Rechtschreibung angepasst.

\subsubsection*{Verwende mehrere Quellen.}

Wichtig ist, sich bei einzelnen Themen nicht auf nur eine Quelle zu
verlassen, sondern mehrere Belege anzuführen. Wenn du also ein Kapitel
über Flow schreibst, dann solltest du dich nicht nur auf
Csikszentmihalyi beziehen, sondern auch auf andere Autor:innen.

\subsubsection*{Du darfst Wikipedia als Quelle verwenden.}

Wikipedia enthält Fehler. Bücher auch. Du darfst Wikipedia als Quelle
verwenden, wenn du die oben beschriebene Regel berücksichtigst und
zusätzlich weitere Quellen (Bücher, Zeitschriftenartikel, ...) anführst.
Genauso darfst du übrigens mit Vorlesungsskripten verfahren. Diese
dürfen verwendet werden, aber auch nicht als einzige Quelle. Insgesamt
gilt die goldene Regel: Immer auf mehrere Werke beziehen! Nur so kannst
du dich absichern!

Weitere Hinweise zur Verwendung von Wikipedia als Quelle lies bitte im
Weblog-Artikel \glqq{}Wikipedia als Quelle\grqq{}\footnote{Christian Spannagel (2010). \emph{Wikipedia als Quelle?}. \url{http://cspannagel.wordpress.com/2010/01/31/wikipedia-als-quelle-2/} (Stand: 31. Januar 2010)} nach.

\subsubsection*{Verweise auf primäre Quellen.}

Gegeben sei folgende Situation: Du liest ein Buch von Schneider. Dieser
belegt eine Aussage mit einem Text von Müller. Wenn du diese Aussage von
Müller in deinem eigenen Text aufnehmen möchtest, so könntest du
versucht sein, diese mit dem Text von Müller zu belegen, ohne diesen
jemals gelesen zu haben. Oder du könntest auf die Idee kommen, die
Aussage mit dem Text von Schneider zu belegen, obwohl dieser nicht der
Urheber der Aussage ist. Widerstehe diesen Versuchungen. Besorge dir den
Originaltext von Müller, lese diesen (zumindest die relevanten Stellen)
und belege dann die Aussage mit Müller. In der Regel erhält man auch für
die eigene Arbeit interessante Anregungen durch das Lesen von
Originalwerken.

Eine Ausnahme kann entstehen, wenn Schneider auf der Basis von anderen
Texten eine neuartige Aussage trifft, deren Urheber er selbst ist. Hier
ein Beispiel: Meier schreibt, dass alle Marsmenschen grün sind. Müller
schreibt, dass alle Jupiterbewohner grün sind. Schmidt schreibt, dass
alle Saturnbewohner grün sind. Schneider schreibt unter Rückgriff auf
Müller, Meier und Schmidt, dass alle Außerirdischen grün sind. Wenn man
nun ein Buch über Außerirdische schreibt und die Aussage treffen möchte,
dass alle grün sind, sollte man dies mit Schneider belegen. Wenn man ein
Buch über den Mars schreibt und aussagen möchte, dass dessen Bewohner
grün sind, dann sollte man dies mit Meier belegen und nicht mit
Schneider. Wenn man nur das Buch von Schneider besitzt, dann sollte man
sich das Buch von Meier besorgen und die relevanten Stellen lesen, bevor
man darauf Bezug nimmt.

Sollte man sich die Originalliteratur trotz respektabler Anstrengung
nicht beschaffen können, dann -- und nur dann! -- darf man
folgendermaßen zitieren: \glqq{}Alle Marsmenschen sind grün.\grqq{} (Meier, 1987, S.
15; zitiert nach Schneider, 2002, S. 47).

\subsubsection*{Verwende in englischsprachigen Texten nur ebensolche Referenzen.}

Solltest du einen englischsprachigen Text verfassen, dann führe
ausschließlich englischsprachige Literatur als Referenzen an. Du ärgerst
dich vermutlich auch, wenn du einen interessanten englischsprachigen
Text liest, in diesem aber überwiegend portugiesische Literaturbelege
enthalten sind.

\subsubsection*{Scheue dich nicht vor englischsprachiger Literatur.}

Im Übrigen gilt, auch wenn du einen deutschen Text verfasst: Scheue dich
nicht vor englischsprachiger Literatur. Wichtige Forschungsarbeiten sind
ausschließlich in Englisch verfasst. Vielleicht sind auch interessante
Ergebnisse in englischsprachiger Literatur, die ganz wichtig für deine
eigene Arbeit sind! Diese Ergebnisse möchtest du doch sicher nicht
ignorieren?

\subsubsection*{Bleibe rechtschaffen.}

Wer zitiert, ohne dies als Zitat zu kennzeichnen, erzeugt ein Plagiat.
Plagiate lassen sich recht leicht ausfindig machen, z.B. indem man
auffällige Sätze in Suchmaschinen eingibt. Für Plagiate gibt es keine
Entschuldigung -- sie führen in der Regel zu einem Nichtbestehen.

Weitere Hinweise zu Plagiaten gibt es auf dem Portal Plagiat\footnote{\url{http://plagiat.htw-berlin.de/}}

\section{Literaturangaben}

Es gibt verschiedene Richtlinien für Literaturangaben. Dieser Text
orientiert sich an den Richtlinien der American Psychological
Association (APA, 2020). Bitte beachte, dass die Formatierung (kursive
Schrift) und die Zeichen zur Trennung der einzelnen Verweisteile
(Punkte, Klammern, \ldots) exakt übernommen werden müssen. Im Folgenden
sind einige Beispiele für Literaturverweise aufgeführt.

\subsubsection*{Buch}

Eco, U. (1993). \emph{Wie man eine wissenschaftliche Abschlussarbeit
schreibt} (6., durchgesehene Auflage der deutschen Ausgabe). Heidelberg:
C. F. Müller.

Horton, W. \& Horton, K. (2003). \emph{E-learning tools and
technologies}. Indianapolis: Wiley.

Niegemann, H. M., Hessel, S., Hochscheid-Mauel, D., Aslanski, K.,
Deimann, M. \& Kreuzberger, G. (2004). \emph{Kompendium E-Learning}.
Berlin, Heidelberg, New York: Springer.

van Merriënboer, J. J. G. (1997). \emph{Training complex cognitive
skills. A four component instructional design model for technical
training}. Englewood Cliffs, NJ: Educational Technology Publications.

\subsubsection*{Herausgeberwerk}

Dempsey, J. V. \& Sales, G. C. (Hrsg.) (1993). \emph{Interactive
instruction and feedback}. Englewood Cliffs, NJ: Educational Technology
Publications.

Salomon, G. (Hrsg.) (1993). \emph{Distributed cognitions. Psychological
and educational considerations}. New York: Cambridge University Press.

\subsubsection*{Journal-Artikel}

Chandler, P. (2004). The crucial role of cognitive processes in the
design of dynamic visualizations. \emph{Learning and Instruction, 14},
353--357.

Deci, E. L. \& Ryan, R. M. (1993). Die Selbstbestimmungstheorie der
Motivation und ihre Bedeutung für die Pädagogik. \emph{Zeitschrift für
Pädagogik, 39}(2), 223--238.

Ertmer, P. A., Evenbeck, E., Cennamo, K. S. \& Lehman, J. D. (1994).
Enhancing self-efficacy for computer technologies through the use of
positive classroom experiences. \emph{Educational Technology Research
and Development, 42}(3), 45--62.

(Bitte beachte: Der Jahrgang (Volume) der Zeitschrift ist kursiv, die
Nummer mit den Klammern nicht! Zwischen Jahrgang und öffnender Klammer
ist auch kein Leerzeichen. Die Seitenzahlen beginnen nicht mit \glqq{}S.\grqq{}.)

\subsubsection*{Kapitel in einem Herausgeberwerk}

Schaumburg, H. (2002). Besseres Lernen durch Computer in der Schule?
Nutzungsbeispiele und Einsatzbedingungen. In L. J. Issing \& P. Klimsa
(Hrsg.), \emph{Information und Lernen mit Multimedia und Internet} (3.,
vollst. überarb. Auflage, S. 335--344). Weinheim: Beltz.

(Bitte beachte: Bei den Herausgeber:innen stehen die abgekürzten Vornamen vor
den Nachnamen. Die Seitenzahlen beginnen hier mit \glqq{}S.\grqq{}.).

\subsubsection*{Artikel in Konferenzbänden (Proceedings)}

Plaisant, C., Kang, H. \& Schneiderman, B. (2003). Helping users get
started with visual interfaces: multi-layered interfaces, integrated
initial guidance and video demonstrations. In \emph{Proceedings of the
10th International Conference on Human-Computer Interaction}, Crete,
Greece, June 22--27, 2003. Hillsdale, NJ: Lawrence Erlbaum Associates.
790--794.

(Bitte beachte: Auch hier steht kein \glqq{}S.\grqq{} vor den Seitenzahlen. Wenn die
Herausgeber:innen bekannt sind, dann sollten sie hinter dem \glqq{}In\grqq{} angeführt
werden, genau wie bei Herausgeberwerken.)

\subsubsection*{Dokumente im Web}

Spannagel, C. (2010). Wikipedia als Quelle?
\url{http://cspannagel.wordpress.com/2010/01/31/wikipedia-als-quelle-2/}
(Stand: 31.1.2010)

(Bitte beachte: Wenn auf der Webseite ein Stand angegeben ist, dann
nenne dieses Datum. Ansonsten nimm das Datum des Tages, an dem du
zuletzt auf die Seite zugegriffen hast.)

\subsubsection*{Abschließende Bemerkungen zum Literaturverzeichnis}

Das Literaturverzeichnis ist alphabetisch aufsteigend sortiert, als
zweites Sortierkriterium wird die Jahreszahl herangezogen (frühere Werke
werden zuerst genannt). Das Literaturverzeichnis enthält genau
diejenigen Elemente, auf die vom Text aus verwiesen wird. Weitere Werke,
die vielleicht im Zusammenhang interessant sind, die aber nicht im Text
referenziert werden, werden dort nicht aufgelistet.


\section{Und schließlich...}

\subsubsection*{Gut geplant ist halb geschrieben.}

Starte früh. Manch einer hat sich schon die Nacht am Tag vor der
Deadline um die Ohren gehauen. Die folgende Reihenfolge sollte dabei
beachtet werden: 1. denken 2. planen 3. schreiben 4. überarbeiten

\subsubsection*{Lies deinen Text vor der Abgabe.}

Es versteht sich von selbst, dass du dich an die aktuell gültigen
Rechtschreib- und Zeichensetzregeln hältst. Lese deinen Text Korrektur
-- am besten zweimal. Und lasse andere Personen deinen Text überprüfen.
Idealerweise bittest du Personen, die fachkundig sind. Es schadet auch
nicht immer, die Resultate der Rechtschreib- und Grammatikprüfung deines
Textverarbeitungsprogramms zu beachten.

\subsubsection*{Beachte das Urheberrecht.}

Du darfst beispielsweise nicht ohne weiteres Bilder aus Büchern
einscannen und in deinem Text verwenden. Grafiken solltest du nochmals
selbst zeichnen und auf das Werk verweisen, an dem du dich orientiert
hast. Wenn du ein Bild nicht selbst reproduzieren kannst (beispielsweise
weil es sich um eine Fotografie eines Marsmenschen handelt und du nicht
extra auf den Mars fliegen möchtest, um ein eigenes Bild zu schießen),
dann kannst du auch versuchen, dir die schriftliche Genehmigung des
Verlags für einen Abdruck des Bildes in deinem eigenen Text zu holen. In
der Regel schreibt man dann einen Hinweis zum Bild, der in etwa wie
folgt lautet: \glqq{}Mit freundlicher Genehmigung des Verlags XYZ\grqq{}.

\subsubsection*{Dieser Text wird ständig weiterentwickelt.}

Lässt dieser Text noch Fragen offen? Hast du Anregungen, Kritik oder
Verbesserungsvorschläge? Ich bin immer dankbar für Rückmeldungen!

Wenn du diesen Text zitierst, gib bitte immer den Stand mit an --
genau so, wie man es bei Dokumenten im Web macht.

\section{Weitere Quellen}

\begin{itemize}
\item Das Projekt helpBW hat einen Online-Kurs zum wissenschaftlichten Schreiben erstellt. Diesen findest du hier: \url{https://helpbw.de/}.
\end{itemize}

\section{Danksagung}

Vielen herzlichen Dank an Christine Bescherer, Raimund Girwidz und
Andreas Zendler für wertvolle Hinweise zu diesem Text.

\section{Literatur}

APA -- American Psychological Association (2020). \emph{Concise Rules of APA Style} (7th Ed.). Washington, D.C.: APA.

Eco, U. (1993). \emph{Wie man eine wissenschaftliche Abschlussarbeit schreibt} (6., durchgesehene Auflage der deutschen Ausgabe). Heidelberg: C. F.
Müller.

\vspace*{10mm}

\printlicense

\printsocials


%
% Ideen für spätere Überarbeitungen
%
% Wheeler's Rules
%
%  Stangl:
%  \href{http://paedpsych.jk.uni-linz.ac.at/INTERNET/ARBEITSBLAETTERORD/PRAESENTATIONORD/DiplomarbeitInhalt.html}{Empfehlungen
%  für die inhaltliche Gestaltung wissenschaftlicher Arbeiten}
%
%  Weber: Der Fünfsatz
%
%  Zeilenabstand
%
%  APA-Style-Buch (neue Ausgabe)
%
%  Walter Simon: GABALS großer Methodenkoffer, Grundlagen der
%  Kommunikation (Kapitel 6,11,12)
%
%  Zahlwörter von eins bis zwölf\ldots{} hundert? usw.
%
%  ebd.
%
%  Hervorhebungen in Zitaten
%
%  Anhang hinter Literaturverzeichnis
%
%  Unterschied zwischen einfachem Beleg und Zitat deutlich hervorheben!
%  D.h. beim einfachen Beleg ist die Aussage selbstständig formuliert!
%  Ansonsten handelt es sich um ein Zitat.
%\item
%  Aussagen anderer immer mit entsprechendem Beleg anführen (man stellt
%  sich ohne Beleg nicht besser -- im Gegenteil: mit Beleg zeigt man,
%  dass man die Literatur kennt).
%\item
%  Publikationsliste eines Autors im Web nach weiterer Literatur
%  durchforsten; Man kann Autoren auch eine Mail schreiben, wenn man
%  anderweitig nicht an einen Artikel herankommt
%\item
%  Tipps zum \glqq{}kreativen Schreiben\grqq{}: irgendjemand (Hemingway?) hat immer
%  mitten im Satz aufgehört zu schreiben, um am nächsten Tag an diesem
%  Gedanken anknüpfen zu können. Das ist zwar etwas extrem. In
%  abgeschwächter Form funktioniert das aber auch, wenn man mitten im
%  Absatz aufhört.
%\item
%  Mottos (rechts eingerückte Zitate zu Beginn eines Kapitels) müssen
%  nicht mit Literatur belegt werden, da es sich nicht um
%  wissenschaftliche Aussagen handelt. Sie sollen lediglich Atmosphäre
%  schaffen. Außerdem wird's albern, wenn man Shakespeare genau belegt.
%\item
%  Hinweise auf weiterführende Literatur
%
%  \begin{itemize}
%  \item
%    Der Dativ ist dem Genitiv sein Tod\ldots{}
%  \item
%    Zwiebelfisch-Newsletter
%  \item
%    Duden
%  \end{itemize}
%\item
%  Ludwig-Reiners-Schema: Je mehr Wörter ein Satz enthält, umso
%  schwieriger ist er verständlich. (Reiners hat ebenfalls Stilkunden
%  geschrieben.)
%\item
%  \href{http://www.montiweb.de/wiki/Abschlu\%C3\%9Farbeiten}{Seite über
%  Abschlussarbeiten auf montiweb}
%\item
%  \href{http://www.sim-md.de/sim_media/downloads/OutlineThesisProposal.pdf}{Richtlinien
%  für Thesis Proposals} - Folien vom FB Informatik der Uni Magdeburg
%\item
%  \href{http://de.wikibooks.org/wiki/Verfassen_wissenschaftlicher_Texte}{Verfassen
%  wissenschaftlicher Texte} auf wikibooks
%\end{itemize}

%\paragraph{Einzuarbeitende Literatur}\label{einzuarbeitende_literatur}}
%
%\begin{itemize}
%\item
%  Karmasin, M. \& Ribing, R. (2009). \emph{Die Gestaltung
%  wissenschaftlicher Arbeiten}. Stuttgart: UTB.
%\item
%  Langer, I., Schulz von Thun, F. \& Tausch, R. (1981). \emph{Sich
%  verständlich ausdrücken}. München: Reinhardt.
%\item
%  Schneider, W. (2001). \emph{Deutsch für Profis. Wege zu gutem Stil}.
%  München: Goldmann.
%\item
%  \href{http://hdz.hdz.tu-dortmund.de/uploads/media/Merkblatt-WissArbeiten-V2.pdf}{Merkblatt
%  der TU Dortmund}
%\item
%  Day, R. A. (1998). \emph{How to write and publish a scientific paper}.
%  Cambridge: Cambridge University Press.
%\end{itemize}
%
\end{document}
