\documentclass{../cssheet}

%--------------------------------------------------------------------------------------------------------------
% Basic meta data
%--------------------------------------------------------------------------------------------------------------

\title{Literaturrecherche}
\author{Prof. Dr. Christian Spannagel}
\date{\today}
\hypersetup{%
    pdfauthor={\theauthor},%
    pdftitle={\thetitle},%
    pdfsubject={Hinweise für die Literaturrecherche},%
    pdfkeywords={literatur, artikel, publikationen, wissenschaft, recherche, datenbanken}
}

%--------------------------------------------------------------------------------------------------------------
% document
%--------------------------------------------------------------------------------------------------------------

\begin{document}
\printtitle
\printdate
\singlespacing

\tableofcontents

\section{Vorwort}

Eine wichtige Voraussetzung beim Schreiben eines wissenschaftlichen Texts ist, dass du dich als Autor:in in dem entsprechenden Inhaltsgebiet sehr gut auskennst. Wenn du noch kein:e Expert:in bist, dann bedeutet das in der Regel, dass du viel lesen musst. In diesem Dokument gibt es einige Tipps, wie du an relevante Literatur kommst.

Die Hinweise in diesem Text beziehen sich auf Literaturrecherchen, die im Rahmen wissenschaftlicher Tätigkeiten in den Feldern der Mathematikdidaktik und der Informatikdidaktik durchgeführt werden. Die Übertragung auf andere Disziplinen ist sicher in irgendeiner Weise eingeschränkt. Nichtsdestotrotz gibt es auch allgemeine Hinweise, die in allen Disziplinen nützlich sein können.

Das Dokument wird ständig überarbeitet. Für Hinweise und Verbesserungsvorschläge bin ich dankbar.

\section{www-Suchmaschinen}
\begin{itemize}
\item In der Regel beginnt man eine Suche in einer traditionellen Suchmaschine wie Google oder Ecosia.
\item Besonders hilfreich für das Finden wissenschaftlicher Texte ist Google Scholar (\url{https://scholar.google.de}). Hier werden im Wesentlichen wissenschaftliche Artikel und Bücher gefunden. Falls die Artikel auch im Web stehen (z.\,B. bei Open-Access-Publikationen), gibt es oftmals einen direkten Link zur PDF-Datei. Außerdem zeigt Google Scholar an, in wie vielen weiteren Publikationen das Werk zitiert wird (\emph{Zitiert von: XX}). Wenn man darauf klickt, erhält man eine Liste dieser Publikationen. Dies ist sehr hilfreich, wenn man nach neuerer Literatur sucht, die sich auf ein bestimmtes Werk bezieht und damit für die eigene Arbeit interessant sein könnte.
\item Selbstverständlich ist auch Google Books (\url{https://books.google.de/}) eine interessante Anlaufstelle.
\end{itemize}
\pagestyle{docstyle}

\section{Fachspezifische Literaturdatenbanken}

Fachspezifische Literaturdatenbanken bieten eine sehr gute Anlaufstelle für die Suche nach disziplinbezogenen Publikationen. Leider wurde die in der Mathematikdidaktik vielfach genutzte Datenbank \emph{MathEduc} vor einiger Zeit abgeschaltet.

\subsubsection*{Mathematikdidaktik}
\begin{itemize}
\item Literatur auf Madipedia: \href{https://madipedia.de/wiki/Mathematikdidaktische_Literatur}{madipedia.de}
\end{itemize}

\subsubsection*{Bildungswissenschaften}
\begin{itemize}
\item ERIC Database - Educational Resources Information Center: \url{https://eric.ed.gov}
\item Fachportal Pädagogik: \url{https://www.fachportal-paedagogik.de/literatur/}
\item Liste auf dem Deutschen Bildungsserver: \href{https://www.bildungsserver.de/Erziehungswissenschaftliche-Literaturdatenbanken-994-de.html}{www.bildungsserver.de}
\end{itemize}

\subsubsection*{Bibliotheken in Heidelberg}
\begin{itemize}
\item PH-Katalog: \href{https://www.ph-heidelberg.de/bibliothek/mediensuche/rechercheangebote/katalog-ph.html}{www.ph-heidelberg.de/bibliothek}
\item Uni-Katalog HEIDI: \url{https://katalog.ub.uni-heidelberg.de/cgi-bin/search.cgi}
\end{itemize}
\subsubsection*{Weitere Links}
\begin{itemize}
\item PubPsych (Psychologie-Datenbank): \url{https://www.pubpsych.de}
\item ReDi - Regionale Datenbank-Information BW: \url{http://www-fr.redi-bw.de}
\end{itemize}

\section{Fachzeitschriften}

Man kann natürlich auch direkt in Fachzeitschriften (online) stöbern. Manche Zeitschriften bieten auch eine Suche an, mit der man nach Schlagworten im jeweiligen Journal suchen kann.
\subsubsection*{Mathematikdidaktik}

\begin{itemize}
\item Journal für Mathematikdidaktik: \url{https://www.springer.com/journal/13138}
\item ZDM -- Mathematics Education: \url{https://www.springer.com/journal/11858}
\item mathematica didactica: \url{http://www.mathematica-didactica.com}
\item Mathematik lehren: \href{https://www.friedrich-verlag.de/mathematik/mathematik-lehren/}{www.friedrich-verlag.de}
\item Mathematik 5-10: \url{https://www.friedrich-verlag.de/mathematik/mathematik-5-10/}
\item Grundschule Mathematik: \href{https://www.friedrich-verlag.de/grundschule/mathematik/grundschule-mathematik/}{www.friedrich-verlag.de}
\item The Electronic Journal of Mathematics \& Technology: \url{https://php.radford.edu/~ejmt}
\item Mathematik differenziert: \url{https://www.westermann.de/zeitschriften/grundschule/mathematik-differenziert}
\item Digital Experiences in Mathematics Education: \url{https://www.springer.com/journal/40751}
\end{itemize}

\subsubsection*{Informatikdidaktik}
\begin{itemize}
\item IBIS (Informatische Bildung in Schulen): \url{https://www.informatischebildung.de/index.php/ibis}
\item LOG IN: \url{https://www.log-in-verlag.de} (bis 2002)
\item Computer Sciencde Education: \url{https://www.tandfonline.com/loi/ncse}
\item ACM Inroads: \url{https://dl.acm.org/magazine/inroads}
\item ACM Transactions on Computing Education: \url{https://dl.acm.org/journal/toce}
\item IEEE Transactions on Education: \url{https://ieeexplore.ieee.org/xpl/RecentIssue.jsp?punumber=13}
\item Education and Information Technologies: \url{https://www.springer.com/journal/10639}
\item Informatics in Education: \url{https://infedu.vu.lt/journal/INFEDU}
\item informatica didactica: \url{https://www.informaticadidactica.de}
\end{itemize}

\subsubsection*{Computers \& Education}
\begin{itemize}
\item Computer + Unterricht: \href{https://www.friedrich-verlag.de/shop/schule-und-unterricht/digitale-schule/fachzeitschriften/computer-unterricht}{www.friedrich-verlag.de}
\item Computers \& Education: \url{https://journals.elsevier.com/computers-and-education}
\end{itemize}

\subsubsection*{Weitere Links}
\begin{itemize}
\item Directory of Open Access Journals: \url{https://doaj.org}
\item Educational Design Research Journal: \url{https://journals.sub.uni-hamburg.de/EDeR/}
\end{itemize}

\section{Weitere Tipps}
\begin{itemize}
\item Frag deine:n betreuende:n Dozent:in nach Literaturtipps.
\item \emph{Exponentielles Wachstum:} Jeder Artikel, den du bereits gefunden hast, enthält ein Literaturverzeichnis mit weiteren interessanten Artikeln, die wiederum ein Literaturverzeichnis mit weiteren interessanten Artikeln haben usw.
\item Auch die Wikipedia enthält oftmals interessante Literaturhinweise. Schau nicht nur in die deutschsprachige, sondern auch in die englischsprachige Wikipedia.
\item Überhaupt gilt: Wissenschaft ist international. Beschränke dich bei deiner Suche auf keinen Fall nur auf deutschsprachige Quellen. Suche also nicht nur nach \glqq{}Informatik Projekt Schule\grqq{}, sondern auch nach \glqq{}computer science project school\grqq{}.
\item Das Projekt helpBW hat einen Online-Kurs zur Literaturrecherche erstellt. Diesen findest du hier: \url{https://helpbw.de/}.

\end{itemize}

\newpage
%\vspace*{10mm}

\printlicense

\printsocials

\end{document}
